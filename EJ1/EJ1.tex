\section*{\color{olive}Ejercicio 1: Medici\'on de curvas caracter\'isticas de diodos}

% \begin{circuitikz}[scale=1.2]\draw
% (0,0) node[ground] {}
% to[V=$e(t)$, *-*] (0,2) to[C=4<\nano\farad>] (2,2)
% to[R, l_=.25<\kilo\ohm>, *-*] (2,0)
% (2,2) to[R=1<\kilo\ohm>] (4,2)
% to[C, l_=2<\nano\farad>, *-*] (4,0)
% (5,0) to[I, i_=$a(t)$, -*] (5,2) -- (4,2)
% (0,0) -- (5,0)
% (0,2) -- (0,3) to[L, l=2<\milli\henry>] (5,3) -- (5,2)
%
% {[anchor=south east] (0,2) node {1} (2,2) node {2} (4,2) node {3}}
% ;\end{circuitikz}

\begin{figure}[h!]
 \begin{center}
    \begin{circuitikz}
    \draw (0,3)
	%(0,0) node[ground] {}
      to[V,v=$V_{cc}$] (0,0) % The voltage source
     (0,3) to[short] (1,3) node[fulldiodeshape]{} 
	(1,3) to[short] (2,3)
      to[R=$R_1$] (2,0) % The resistor
	to[short] (2,0)
	to[short] (1,0) node[ground]{}
	(1,0) to[short] (0,0);
    \end{circuitikz}
    \caption{Circuito empleado para medir la curva caracter\'istica de un diodo.}
\end{center}
\end{figure}


\begin{circuitikz}[american]
 \draw (0,0) 
to[isource, l=$I_0$]  	%Vertical
(0,3) -- (2,3)				%linea horizontal arriba entre las dos primeras ramas \\
to[R=$R_1$] 				%Vertical
(2,0) -- (0,0);				%linea horizontal abajo entre dos primeras ramas \\
 \draw (2,3) -- (4,3) 
to[R=$R_2$]				%Vertical
(4,0) -- (2,0);
\draw (2,0) node[ground] {};
 \end{circuitikz}

\begin{circuitikz}[american]
 \draw (0,0) to[isource, l=$I_0$] (0,3)
 to[short, -*, i=$I_0$] (2,3)
 to[R=$R_1$, i>_=$i_1$] (2,0) -- (0,0);
 \draw (2,3) -- (4,3)
 to[R=$R_2$, i>_=$i_2$]
 (4,0) to[short, -*] (2,0);













 \end{circuitikz}