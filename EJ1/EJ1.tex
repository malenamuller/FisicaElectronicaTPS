\section*{\color{olive}Ejercicio 1: Medici\'on de curvas caracter\'isticas de diodos}

A continuaci\'on se presentan los circuitos empleados para medir la curva caracter\'istica (corriente en funci\'on de la tensi\'on) de un diodo rectificador, de un diodo zener y de un LED. En los tres casos se coloca una resistencia en serie de $1k\Omega$, con la finaldidad de limitar la corriente que circula por cada uno de ellos.

\begin{figure}[h!]
 \begin{center}
    \begin{circuitikz}
    \draw (0,3)
	%(0,0) node[ground] {}
      to[V,v=$V_{cc}$] (0,0) % The voltage source
	(0,3) to[Do] (2,3)
%  (0,3) to[short] (1,3) node[fulldiodeshape]{} 
%	(1,3) to[short] (2,3)
      to[R=$R_1$] (2,0) % The resistor
	to[short] (2,0)
	to[short] (1,0) node[ground]{}
	(1,0) to[short] (0,0);

    \draw (6,3)
	%(0,0) node[ground] {}
      to[V,v=$V_{cc}$] (6,0) % The voltage source
	(6,3) to[zDo] (8,3)
%  (0,3) to[short] (1,3) node[fulldiodeshape]{} 
%	(1,3) to[short] (2,3)
      to[R=$R_1$] (8,0) % The resistor
	to[short] (8,0)
	to[short] (7,0) node[ground]{}
	(7,0) to[short] (6,0);

    \draw (12,3)
	%(0,0) node[ground] {}
      to[V,v=$V_{cc}$] (12,0) % The voltage source
	(12,3) to[leDo] (14,3)
%  (0,3) to[short] (1,3) node[fulldiodeshape]{} 
%	(1,3) to[short] (2,3)
      to[R=$R_1$] (14,0) % The resistor
	to[short] (14,0)
	to[short] (13,0) node[ground]{}
	(14,0) to[short] (12,0);
    \end{circuitikz}

    \caption{Circuitos empleados para medir la curva caracter\'istica de un diodo rectificador, de un diodo zener y de un LED; respectivamente.}
\end{center}
\end{figure}

%%% diodo rectificador
\subsection*{\color{orange}Diodo rectificador}

A continuaci\'on se presentan los gr\'aficos obtenidos de la corriente vs. tensi\'on para el caso del diodo rectificador 1N4148. En la figura \ref{med1b} se puede ver, a la izquierda, los datos obtenidos del osciloscopio mediante un CSV y a la derecha el gr\'afico de la corriente en funci\'on de la tensi\'on a partir de las tensiones medidas con osciloscopio y del valor conocido de la resistencia (azul). En ese mismo gr\'afico tambi\'en se puede ver la corriente vs. tensi\'on simulada (narajna).

\begin{figure}[H]
\centering
\includegraphics[scale=0.5]{../EJ1/DiodoRectificador/datosOsciloscopio}
\includegraphics[scale=0.5]{../EJ1/DiodoRectificador/rectificadorSuperpos}
\caption{Medici\'on de la corriente vs. tensi\'on del diodo rectificador: Datos obtenidos y datos procesados vs. simulados; respectivamente.}
\label{med1b}
\end{figure}

\begin{figure}[H]
\centering
\includegraphics[scale=0.8]{../EJ1/DiodoRectificador/corrienteDiodoDatasheet}
\caption{Corriente vs. tensi\'on del diodo rectificador obtenida de la hoja de datos.}
\label{med1c}
\end{figure}

De la figura \ref{med1c} obtenida de la hoja de datos del diodo 1N4148, a nuestro caso corresponde la curva (2).

Se puede ver que la curva caracter\'istica del diodo obtenida con la medici\'on es muy similar tanto a la simulaci\'on, como a la curva de la hoja de datos de dicho componente. En el caso medido se observa que la corriente comienza a aumentar a una tensi\'on levemente mayor que para el caso simulado.

%%% diodo zener
\subsection*{\color{orange}Diodo zener}

A continuaci\'on, en la figura \ref{med2b} a la derecha, se presenta un gr\'afico con la simulaci\'on de la corriente en funci\'on de la tensi\'on correspondiente al diodo zener en naranja, y lo obtenido a partir de las mediciones del osciloscopio, en azul. Para obtener este \'ultimo se procedi\'o de la misma manera que para obtener el correspondiente al diodo rectificador (utilizando los datos obtenidos de tensiones en el circuito, que se pueden ver en el gr\'afico de la izquierda de la figura \ref{med2b}).


\begin{figure}[H]
\centering
\includegraphics[scale=0.5]{../EJ1/DiodoZener/datosOsciloscopioZener}
\includegraphics[scale=0.5]{../EJ1/DiodoZener/zenerSuperpos}
\caption{Medici\'on de la corriente vs. tensi\'on del diodo zener: Datos obtenidos y datos procesados vs. simulación; respectivamente.}
\label{med2b}
\end{figure}

%\begin{figure}[!ht]
%\centering
%\includegraphics[scale=0.52]{../EJ1/DiodoZener/corrienteDiodoDatasheet}
%\caption{Corriente vs. tensi\'on del diodo zener obtenida de la hoja de datos.}
%\label{med2c}
%\end{figure}

En el gr\'afico de la derecha en la figura \ref{med2b} se puede ver que lo medido y la simulaci\'on son pr\'acticamente iguales, con una diferencia pequeña de tensi\'on a la que comienza a circular corriente en inversa. En la medici\'on se observa un encendido mas temprano (a una tensi\'on en inversa de menor m\'odulo que en el caso de la simulaci\'on).

%%% led
\subsection*{\color{orange}LED}

En la figura \ref{med3b} se observa, como en los dos casos previos, a la izquierda las tensiones obtenidas del osciloscopio para obtener luego la curva de corriente del led en funci\'on de la tensi\'on, que puede verse en azul en el gr\'afico de la derecha. Ese mismo gr\'afico contiene la curva de la simulaci\'on del led rojo. La figura \ref{med3c} presenta la curva mencionada, pero proveniente de la hoja de datos.

\begin{figure}[H]
\centering
\includegraphics[scale=0.5]{../EJ1/LED/datosOsciloscopioLED}
\includegraphics[scale=0.5]{../EJ1/LED/ledSuperpos}
\caption{Medici\'on de la corriente vs. tensi\'on del LED: Datos obtenidos y datos procesados vs simulaci\'on; respectivamente.}
\label{med3b}
\end{figure}

\begin{figure}[!ht]
\centering
\includegraphics[scale=0.9]{../EJ1/LED/LedDataSheet}
\caption{Corriente vs. tensi\'on del LED obtenida de la hoja de datos.}
\label{med3c}
\end{figure}

La simulaci\'on dio pr\'acticamente igual a la medici\'on de la corriene en el led, en funci\'on de la tensi\'on.







