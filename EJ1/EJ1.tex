\section*{\color{olive}Ejercicio 1: Medici\'on de curvas caracter\'isticas de diodos}



\begin{figure}[h!]
 \begin{center}
    \begin{circuitikz}
    \draw (0,3)
	%(0,0) node[ground] {}
      to[V,v=$V_{cc}$] (0,0) % The voltage source
(0,3) to[Do] (2,3)
%  (0,3) to[short] (1,3) node[fulldiodeshape]{} 
%	(1,3) to[short] (2,3)
      to[R=$R_1$] (2,0) % The resistor
	to[short] (2,0)
	to[short] (1,0) node[ground]{}
	(1,0) to[short] (0,0);
 

    \draw (6,3)
	%(0,0) node[ground] {}
      to[V,v=$V_{cc}$] (6,0) % The voltage source
(6,3) to[zDo] (8,3)
%  (0,3) to[short] (1,3) node[fulldiodeshape]{} 
%	(1,3) to[short] (2,3)
      to[R=$R_1$] (8,0) % The resistor
	to[short] (8,0)
	to[short] (7,0) node[ground]{}
	(7,0) to[short] (6,0);

    \draw (12,3)
	%(0,0) node[ground] {}
      to[V,v=$V_{cc}$] (12,0) % The voltage source
	(12,3) to[leDo] (14,3)
%  (0,3) to[short] (1,3) node[fulldiodeshape]{} 
%	(1,3) to[short] (2,3)
      to[R=$R_1$] (14,0) % The resistor
	to[short] (14,0)
	to[short] (13,0) node[ground]{}
	(14,0) to[short] (12,0);
    \end{circuitikz}

    \caption{\color{cyan}Circuito empleado para medir la curva caracter\'istica de un diodo rectificador.}
\end{center}
\end{figure}

%%%%%%%%%%%%%%%%%%%%%%%%%%%%%%%%%%%%%%%%%%%%%%





