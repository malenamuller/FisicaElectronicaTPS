\section*{\color{olive}Ejercicio 2: C\'alculo y simulaci\'on de una funci\'on transferencia de tensi\'on}

\begin{figure}[H] %!ht
 \begin{center}
    \begin{circuitikz}[american]
    \draw (0,3) to[sV,v=$V_{in}$] (0,0) % The voltage source
(0,3)  to[short] (1,3) to[R, l2_=$R_G$ and $50\Omega$] 
(2,3)  to[short] (3,3) to[C=$C_{in}$] (4,3) to[short] (5.5,3) 
%to[R=$R_G$]  (2,3)  to[short] (3,3) to[C=$c_{in}$] (4,3) to[short] (5.5,3) 
(6,3) node[npn]{}
(0,0) to[short] (6,0) to[R=$R_E$] (6,2.5)
(4.5,0) to[R=$R_2$] (4.5,3) to[R=$R_1$] (4.5,6) to[short] (6,6) to[R=$R_C$] (6,3.5)
 to[C=$C_{out}$] (9,3.5) node[anchor=west] {OUT} (9,3.5)
 to[R=$R_L$] (9,0) to[short] (6,0)
(7.5,0) to[pC=$C_E$] (7.5,2.5) to[short] (6,2.5)
(4.5,6)  to[V, v=$V_{cc}$] (0,6) node[ground]{}
(2.5,3) node[anchor=south] {IN} 
(4.5,0) node[ground]{};
    \end{circuitikz}
    \caption{Circuito con un transistor NPN BC547B, para el cual se obtiene la funci\'on transferencia.}
	\label{circ2}
\end{center}
\end{figure}


Siendo
%\begin{itemize}
%\item $ R_1 = 100k\Omega$
%\item $ R_2 = 27k\Omega$
%\item $ R_C = 11.2k\Omega$
%\item $ R_E = 3k\Omega$
%\item $ R_L = 10k\Omega$
%\item $ C_{in} = 20nF$
%\item $ C_{out} = 10nF$
%\item $ C_E = 2\mu F$
%\end{itemize}

\begin{multicols}{4}
\begin{itemize}
\item $ R_1 = 100k\Omega$
\item $ R_2 = 27k\Omega$
\end{itemize}
\columnbreak
\begin{itemize}
\item $ R_C = 11.2k\Omega$
\item $ R_L = 10k\Omega$
\end{itemize}
\columnbreak
\begin{itemize}
\item $ R_E = 3k\Omega$
\item $ C_E = 2\mu F$
\end{itemize}
\columnbreak
\begin{itemize}
\item $ C_{in} = 20nF$
\item $ C_{out} = 10nF$
\end{itemize}
\end{multicols}


\subsection*{\color{orange} C\'alculo de la funci\'on transferencia de tensi\'on}

Para calcular la funci\'on transferencia de tensi\'on del circuito \ref{circ2}, se utiliza el modelo h\'ibrido $\pi$ como circuito equivalente del transistor NPN en peque\~na se\~nal, pasivando la fuente de tensi\'on cont\'inua. Adem\'as, a frecuencias medias se considera que los capacitores se comportan como cortocircuitos. El siguiente circuito es el equivalente correspondiente al circuito \ref{circ2}:

\begin{figure}[H]%!ht
 \begin{center}
    \begin{circuitikz}[american]
    \draw (0,1.5) to[sV,v=$V_{in}$] (0,0) % The voltage source
(0,1.5) to[R=$R_G$] (0,3)
(2,0) to[R=$R_1$] (2,3)
(4,0) to[R=$R_2$] (4,3)
(6,0) to[R=$R_{\pi}$] (6,3)
(8,3) to[cI=$I_1$] (8,0)
(10,0) to[R=$R_0$] (10,3)
(12,0) to[R=$R_C$] (12,3)
(14,0) to[R=$R_L$] (14,3)
	
(0,0) to[short] (14,0)
(0,3) to[short] (6,3)
(8,3) to[short] (14,3)
(7,0) node[ground]{}
(1,3) node[anchor=south] {IN} 
(14,3) node[anchor=west] {OUT};
    \end{circuitikz}
    \caption{Circuito equivalente empleado para el c\'alculo de la funci\'on transferencia de tensi\'on.}
	\label{circ22}
\end{center}
\end{figure}

En el gr\'afico anterior, siendo:
$$I_1 = \beta \cdot i_b$$
con $i_b$ la corriente que circula por la resistencia denominada $R_{\pi}$.

A partir del circuito \ref{circ22}, surge que:
\begin{equation}\frac{V_{OUT}}{V_{IN}} = - \frac{\left( R_0 // R_C // R_L\right) \beta}{ R_{\pi}} = - \frac{\beta \cdot (R_0 \cdot \frac{R_C R_L}{R_C + R_L})}{R_{\pi} \cdot (R_0 + \frac{R_C R_L}{R_C + R_L})}  \label{ec1} \end{equation}

A partir de la simulaci\'on, calculamos la impedancia de salida del circuito de la figura \ref{circ2} y obtuvimos el valor de $R_0$ del transistor NPN BC547B:\\
$R_0 = 103 \cdot k\Omega$\\
A su vez, usando el resto de los valores con los que est\'a modelada la simulaci\'on:\\
$R_{\pi} = 12,3 k\Omega$\\
$\beta = 294$\\

Con los valores anteriores, reemplazando en la ecuaci\'on \ref{ec1} se obtiene que:
$$ \frac{V_{OUT}}{V_{IN}} = -120,116 $$

Entonces:

\begin{equation} |\frac{V_{OUT}}{V_{IN}} |_{db} = 41,592 dB \label{ec2} \end{equation}

\subsection*{\color{orange} Simulaci\'on de la funci\'on transferencia de tensi\'on}


\begin{figure}[H] %!ht
\centering
\includegraphics[scale=0.45]{../EJ2/rtaenfrec}
\caption{Simulaci\'on del circuito \ref{circ2}}
\label{simEj2}
\end{figure}

En la simulaci\'on de la figura \ref{simEj2}, la l\'inea oscura corresponde al m\'odulo de la funci\'on transferencia, mientras que la clara corresponde a su fase. 

\subsection*{\color{orange} Comparaci\'on entre el c\'alculo y la simulaci\'on de la funci\'on transferencia}

A partir de la l\'inea oscura del diagrama de Bode de la figura \ref{simEj2} (que corresponde al m\'odulo de la ganancia en dB), se puede ver que a frecuencias medias (entre 10kHz y 1MHz, aproximadamente) coincide con  el valor obtenido en la ecuaci\'on \ref{ec2}. Esta comparaci\'on la hacemos a frecuencias medias, ya que el c\'alculo de la funci\'on transferencia, con el cu\'al se obtuvo que la ganancia es de 41,592 dB, surge de modelar un circuito equivalente del circuito \ref{circ2} para pequeña señal y para dichas frecuencias, donde los capacitores se comportan como un cable.




















