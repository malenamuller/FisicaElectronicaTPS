\section*{\color{olive}Ejercicio 2: C\'alculo y simulaci\'on de una funci\'on transferencia de tensi\'on}

\begin{figure}[!ht]
 \begin{center}
    \begin{circuitikz}[american]
    \draw (0,3) to[V,v=$V_{in}$] (0,0) % The voltage source
(0,3)  to[short] (1,3) to[R, l2_=$R_G$ and $50\Omega$] 
(2,3)  to[short] (3,3) to[C=$C_{in}$] (4,3) to[short] (5.5,3) 
%to[R=$R_G$]  (2,3)  to[short] (3,3) to[C=$c_{in}$] (4,3) to[short] (5.5,3) 
(6,3) node[npn]{}
(0,0) to[short] (6,0) to[R=$R_E$] (6,2.5)
(4.5,0) to[R=$R_2$] (4.5,3) to[R=$R_1$] (4.5,6) to[short] (6,6) to[R=$R_C$] (6,3.5)
 to[C=$C_{out}$] (9,3.5) node[anchor=west] {OUT} (9,3.5)
 to[R=$R_L$] (9,0) to[short] (6,0)
(7.5,0) to[pC=$C_E$] (7.5,2.5) to[short] (6,2.5)
(4.5,6)  to[V, v=$V_{cc}$] (0,6) node[ground]{}
(2.5,3) node[anchor=south] {IN} 
(4.5,0) node[ground]{}
;
    \end{circuitikz}
    \caption{\color{cyan}Circuito empleado para medir la curva caracter\'istica de un diodo.}
\end{center}
\end{figure}


Siendo
\begin{itemize}
\item $ R_1 = 100k\Omega$
\item $ R_2 = 27k\Omega$
\item $ R_C = 11.2k\Omega$
\item $ R_E = 3k\Omega$
\item $ R_L = 10k\Omega$
\item $ C_{in} = 20nF$
\item $ C_{out} = 10nF$
\item $ C_E = 2\mu F$
\end{itemize}