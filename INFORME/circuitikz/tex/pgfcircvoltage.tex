% Copyright 2007-2009 by Massimo Redaelli
%
% This file may be distributed and/or modified
%
% 1. under the LaTeX Project Public License and/or
% 2. under the GNU Public License.
%
% See the files gpl-3.0_license.txt and lppl-1-3c_license.txt for more details.

%%%%%%%%%%%%%%%%%%%%%%%%%%%%%%%%%%%%%%%%%%
%%  Voltage management


\ctikzset{v^>/.style = {
        v = #1,
        \circuitikzbasekey/bipole/voltage/direction = forward,
        \circuitikzbasekey/bipole/voltage/position = above
    }
}

\ctikzset{v^</.style = {
        v = #1,
        \circuitikzbasekey/bipole/voltage/direction = backward,
        \circuitikzbasekey/bipole/voltage/position = above
    }
}

\ctikzset{v_>/.style = {
        v = #1,
        \circuitikzbasekey/bipole/voltage/direction = forward,
        \circuitikzbasekey/bipole/voltage/position = below
    }
}

\ctikzset{v_</.style = {
        v = #1,
        \circuitikzbasekey/bipole/voltage/direction = backward,
        \circuitikzbasekey/bipole/voltage/position = below
    }
}

\ctikzset{v_/.style = {v = #1, \circuitikzbasekey/bipole/voltage/position = below} }
\ctikzset{v^/.style = {v = #1, \circuitikzbasekey/bipole/voltage/position  = above} }
\ctikzset{v>/.style = {v = #1, \circuitikzbasekey/bipole/voltage/direction = forward} }
\ctikzset{v</.style = {v = #1, \circuitikzbasekey/bipole/voltage/direction = backward} }

% Default position varies whether the component is a voltage source
% or not
\ctikzset{v/.code = {
        \ifpgf@circuit@bipole@isvoltage
            \pgfkeys{\circuitikzbasekey/bipole/voltage/position=above,
            \circuitikzbasekey/bipole/voltage/direction=forward}
        \else
            \ifpgf@circ@oldvoltagedirection
                \pgfkeys{\circuitikzbasekey/bipole/voltage/position=below,
                \circuitikzbasekey/bipole/voltage/direction=backward}
            \else
                \pgfkeys{\circuitikzbasekey/bipole/voltage/position=below,
                \circuitikzbasekey/bipole/voltage/direction=forward}
            \fi
        \fi
        \ifpgf@circ@oldvoltagedirection
            \ifpgf@circuit@bipole@iscurrent\ifpgf@circ@fixbatteries
                \pgfkeys{\circuitikzbasekey/bipole/voltage/position=below,
                \circuitikzbasekey/bipole/voltage/direction=forward}
        \fi\fi
        \else
        \ifpgf@circuit@bipole@iscurrent
            \ifpgf@circuit@bipole@current@backward
                \pgfkeys{\circuitikzbasekey/bipole/voltage/position=below,
                \circuitikzbasekey/bipole/voltage/direction=forward}
            \else
                \pgfkeys{\circuitikzbasekey/bipole/voltage/position=below,
                \circuitikzbasekey/bipole/voltage/direction=backward}
            \fi\fi\fi
            \pgfkeys{/tikz/circuitikz/bipole/voltage/label/name=#1}
            \ctikzsetvalof{bipole/voltage/label/unit}{}
            \ifpgf@circ@siunitx
                \pgf@circ@handleSI{#1}
                \ifpgf@circ@siunitx@res
                    \edef\pgf@temp{\pgf@circ@handleSI@val}
                    \pgfkeyslet{/tikz/circuitikz/bipole/voltage/label/name}{\pgf@temp}
                    \edef\pgf@temp{\pgf@circ@handleSI@unit}
                    \pgfkeyslet{/tikz/circuitikz/bipole/voltage/label/unit}{\pgf@temp}
                \else
            \fi
            \else
        \fi
    }
}

% american voltage font selection and symbol definition
% the default font command is {} --- nothing
\def\pgf@circ@avfont{\ctikzvalof{voltage/american font}}
%
% plus and minus symbols (default is $+$ and $-$, see pgfcirc.defines.tex)
% notice that the double braces are needed  to be able
% to use \boldmath in the font (although it is semi-deprecated...)
%
\def\pgf@circ@avplus{\ctikzvalof{voltage/american plus}}
\def\pgf@circ@avminus{\ctikzvalof{voltage/american minus}}

%% Output routine for generic bipoles

\def\pgf@circ@drawvoltagegeneric{
    \pgfextra{
        \edef\pgf@temp{/tikz/circuitikz/bipoles/\pgfkeysvalueof{/tikz/circuitikz/bipole/kind}/voltage/straight label distance}
        \pgfkeysifdefined{\pgf@temp}
        {
            \edef\partheight{\ctikzvalof{bipoles/\pgfkeysvalueof{/tikz/circuitikz/bipole/kind}/voltage/straight label distance}}
            \edef\tmpdistfromline{(\partheight\pgf@circ@Rlen)}
        }
        {
            \pgfkeysifdefined{/tikz/circuitikz/bipoles/voltage/straight label distance}
            {
                \edef\partheight{\ctikzvalof{bipoles/voltage/straight label distance}}
                \edef\tmpdistfromline{(\partheight\pgf@circ@Rlen)}
            }
            {%calculate default value from part height
                \edef\pgf@temp{/tikz/circuitikz/bipoles/\pgfkeysvalueof{/tikz/circuitikz/bipole/kind}/height}
                \pgfkeysifdefined{\pgf@temp}
                {
                    \edef\partheight{0.5*\ctikzvalof{bipoles/\pgfkeysvalueof{/tikz/circuitikz/bipole/kind}/height}}
                    \edef\tmpdistfromline{(\partheight\pgf@circ@Rlen+0.2\pgf@circ@Rlen)}
                }
                {
                    \edef\tmpdistfromline{(.5\pgf@circ@Rlen)} %fallback to fixed value
                }
            }
        }
        \ifnum \ctikzvalof{mirror value}=-1
        \ifpgf@circuit@bipole@inverted
            \ifpgf@circuit@bipole@voltage@straight
                \def\distfromline{\tmpdistfromline}
            \else
                \def\distfromline{\ctikzvalof{voltage/distance from line}\pgf@circ@Rlen}
            \fi
            \else
            \ifpgf@circuit@bipole@voltage@straight
                \def\distfromline{-\tmpdistfromline}
            \else
                \def\distfromline{-\ctikzvalof{voltage/distance from line}\pgf@circ@Rlen}
            \fi
        \fi
        \else
            \ifpgf@circuit@bipole@inverted
                \ifpgf@circuit@bipole@voltage@straight
                    \def\distfromline{-\tmpdistfromline}
                \else
                    \def\distfromline{-\ctikzvalof{voltage/distance from line}\pgf@circ@Rlen}
                \fi
                \else
                \ifpgf@circuit@bipole@voltage@straight
                    \def\distfromline{\tmpdistfromline}
                \else
                    \def\distfromline{\ctikzvalof{voltage/distance from line}\pgf@circ@Rlen}
                \fi
            \fi
        \fi
        \ifpgf@circuit@bipole@voltage@below
            \def\pgf@circ@voltage@angle{90}
        \else
            \def\pgf@circ@voltage@angle{-90}
        \fi
        \edef\pgf@temp{/tikz/circuitikz/bipoles/\pgfkeysvalueof{/tikz/circuitikz/bipole/kind}/voltage/distance from node}
        \pgfkeysifdefined{\pgf@temp}
        { \edef\distacefromnode{\ctikzvalof{bipoles/\pgfkeysvalueof{/tikz/circuitikz/bipole/kind}/voltage/distance from node}} }
        { \edef\distacefromnode{\ctikzvalof{voltage/distance from node}} }
        \edef\pgf@temp{/tikz/circuitikz/bipoles/\pgfkeysvalueof{/tikz/circuitikz/bipole/kind}/voltage/bump b}
        \pgfkeysifdefined{\pgf@temp}
        { \edef\bumpb{\ctikzvalof{bipoles/\pgfkeysvalueof{/tikz/circuitikz/bipole/kind}/voltage/bump b}} }
        { \edef\bumpb{\ctikzvalof{voltage/bump b}} }
        \edef\shiftv{\ctikzvalof{voltage/shift}}
        \pgfmathsetmacro{\bumpb}{\bumpb + \shiftv} %% adjust the bump is shift
        \ifpgf@circuit@bipole@inverted
            \pgfmathsetmacro{\shiftv}{-\shiftv}
        \fi
        \ifnum \ctikzvalof{mirror value} = -1
            \pgfmathsetmacro{\shiftv}{-\shiftv}
        \fi
    }
    % %\pgf@circ@Rlen/\pgfkeysvalueof{/tikz/circuitikz/current arrow scale} is equal to the length of the currarrow
    coordinate (pgfcirc@midtmp) at ($(\tikztostart) ! \pgf@circ@Rlen/\pgfkeysvalueof{/tikz/circuitikz/current arrow scale} ! (anchorstartnode)$) %absolute move, minimum space is length of arrowhead
    coordinate (pgfcirc@midtmp) at ($(pgfcirc@midtmp) ! \distacefromnode ! (anchorstartnode)$)

    coordinate (pgfcirc@Vfrom) at ($(pgfcirc@midtmp) ! -\distfromline ! \pgf@circ@voltage@angle:(anchorstartnode)$)
    coordinate (pgfcirc@midtmp) at ($(\tikztotarget) ! \pgf@circ@Rlen/\pgfkeysvalueof{/tikz/circuitikz/current arrow scale} ! (anchorendnode)$)%absolute move, minimum space is length of arrowhead
    coordinate (pgfcirc@midtmp) at ($(pgfcirc@midtmp) ! \distacefromnode ! (anchorendnode)$)

    coordinate (pgfcirc@Vto) at ($(pgfcirc@midtmp) ! \distfromline ! \pgf@circ@voltage@angle : (anchorendnode)$)

    \ifpgf@circuit@bipole@voltage@below
        coordinate (pgfcirc@Vto) at ($(pgfcirc@Vto) ! \shiftv!90 :  (anchorendnode)$)
        coordinate (pgfcirc@Vfrom) at ($(pgfcirc@Vfrom) ! \shiftv!-90 :  (anchorstartnode)$)
        coordinate (pgfcirc@Vcont1) at ($(\ctikzvalof{bipole/name}.center) ! \bumpb ! (\ctikzvalof{bipole/name}.-110)$)
        coordinate (pgfcirc@Vcont2) at ($(\ctikzvalof{bipole/name}.center) ! \bumpb ! (\ctikzvalof{bipole/name}.-70)$)
    \else
        coordinate (pgfcirc@Vto) at ($(pgfcirc@Vto) ! -\shiftv!90 :  (anchorendnode)$)
        coordinate (pgfcirc@Vfrom) at ($(pgfcirc@Vfrom) ! -\shiftv!-90 :  (anchorstartnode)$)
        coordinate (pgfcirc@Vcont1) at ($(\ctikzvalof{bipole/name}.center) ! \bumpb ! (\ctikzvalof{bipole/name}.110)$)
        coordinate (pgfcirc@Vcont2) at ($(\ctikzvalof{bipole/name}.center) ! \bumpb ! (\ctikzvalof{bipole/name}.70)$)
    \fi

    \ifpgf@circuit@europeanvoltage
        \ifpgf@circuit@bipole@voltage@straight
            \ifpgf@circuit@bipole@voltage@backward
                (pgfcirc@Vto) --(pgfcirc@Vfrom) node[currarrow, sloped,  allow upside down, pos=1,anchor=tip] {}
            \else
                (pgfcirc@Vfrom) --(pgfcirc@Vto) node[currarrow, sloped,  allow upside down, pos=1,anchor=tip] {}
            \fi
            \else
            \ifpgf@circuit@bipole@voltage@backward
                (pgfcirc@Vto) .. controls (pgfcirc@Vcont2)  and (pgfcirc@Vcont1) ..
                node[currarrow, sloped,  allow upside down, pos=1] {}
                (pgfcirc@Vfrom)
            \else
                (pgfcirc@Vfrom) .. controls (pgfcirc@Vcont1)  and (pgfcirc@Vcont2) ..
                node[currarrow, sloped,  allow upside down, pos=1] {}
                (pgfcirc@Vto)
            \fi
        \fi
        \else
        \ifpgf@circuit@bipole@voltage@backward
            \ifpgf@circ@oldvoltagedirection
                (pgfcirc@Vfrom) node[inner sep=0, node font=\pgf@circ@avfont,
                    anchor=\pgf@circ@bipole@voltage@label@anchor]{\pgf@circ@avplus}
                (pgfcirc@Vto) node[inner sep=0, node font=\pgf@circ@avfont,
                    anchor=\pgf@circ@bipole@voltage@label@anchor]{\pgf@circ@avminus}
            \else
                (pgfcirc@Vfrom) node[inner sep=0, node font=\pgf@circ@avfont,
                    anchor=\pgf@circ@bipole@voltage@label@anchor]{\pgf@circ@avminus}
                (pgfcirc@Vto) node[inner sep=0, node font=\pgf@circ@avfont,
                    anchor=\pgf@circ@bipole@voltage@label@anchor]{\pgf@circ@avplus}
            \fi
            \else
            \ifpgf@circ@oldvoltagedirection
                (pgfcirc@Vfrom) node[inner sep=0, node font=\pgf@circ@avfont,
                    anchor=\pgf@circ@bipole@voltage@label@anchor]{\pgf@circ@avminus}
                (pgfcirc@Vto) node[inner sep=0, node font=\pgf@circ@avfont,
                    anchor=\pgf@circ@bipole@voltage@label@anchor]{\pgf@circ@avplus}
            \else
                (pgfcirc@Vfrom) node[inner sep=0, node font=\pgf@circ@avfont,
                    anchor=\pgf@circ@bipole@voltage@label@anchor]{\pgf@circ@avplus}
                (pgfcirc@Vto) node[inner sep=0, node font=\pgf@circ@avfont,
                    anchor=\pgf@circ@bipole@voltage@label@anchor]{\pgf@circ@avminus}
            \fi
        \fi
    \fi
}

%% Output routine for voltage sources
\def\pgf@circ@drawvoltagegenerator{
    % the following is affected indirectly by voltage/shift, you can move the arrow with voltage/bump a.
    % it's not perfect, but I can't find the way to do it correctly...
    \pgfextra{
        \edef\shiftv{\ctikzvalof{voltage/shift}}
        \edef\bumpa{\ctikzvalof{voltage/bump a}}
        \pgfmathsetmacro{\bumpaplus}{\bumpa + \shiftv}
    }
    \ifpgf@circuit@bipole@voltage@below
        coordinate (pgfcirc@Vfrom) at ($(\ctikzvalof{bipole/name}.center) ! \bumpaplus ! (\ctikzvalof{bipole/name}.-120)$)
        coordinate (pgfcirc@Vto) at ($(\ctikzvalof{bipole/name}.center) ! \bumpaplus ! (\ctikzvalof{bipole/name}.-60)$)
    \else
        coordinate (pgfcirc@Vfrom) at ($ (\ctikzvalof{bipole/name}.center) ! \bumpaplus ! (\ctikzvalof{bipole/name}.120)$)
        coordinate (pgfcirc@Vto) at ($ (\ctikzvalof{bipole/name}.center) ! \bumpaplus ! (\ctikzvalof{bipole/name}.60)$)
    \fi
    \ifpgf@circuit@europeanvoltage
        \ifpgf@circuit@bipole@voltage@backward
            (pgfcirc@Vto)  -- node[currarrow, sloped,  allow upside down, pos=1] {} (pgfcirc@Vfrom)
        \else
            (pgfcirc@Vfrom)  -- node[currarrow, sloped,  allow upside down, pos=1] {} (pgfcirc@Vto)
        \fi
        \else% american voltage
        \ifpgf@circuit@bipole@voltageoutsideofsymbol
            % if it is a battery, must put + and -

            \ifpgf@circ@fixbatteries
                \ifpgf@circuit@bipole@voltage@backward
                    (pgfcirc@Vfrom)  node[node font=\pgf@circ@avfont] {\pgf@circ@avplus}
                    (pgfcirc@Vto) node[node font=\pgf@circ@avfont] {\pgf@circ@avminus}
                \else
                    (pgfcirc@Vfrom)  node[node font=\pgf@circ@avfont] {\pgf@circ@avminus}
                    (pgfcirc@Vto) node[node font=\pgf@circ@avfont] {\pgf@circ@avplus}
                \fi
                \else
                \ifpgf@circuit@bipole@voltage@backward
                    (pgfcirc@Vfrom)  node[node font=\pgf@circ@avfont] {\pgf@circ@avminus}
                    (pgfcirc@Vto) node[node font=\pgf@circ@avfont] {\pgf@circ@avplus}
                \else
                    (pgfcirc@Vfrom)  node[node font=\pgf@circ@avfont] {\pgf@circ@avplus}
                    (pgfcirc@Vto) node[node font=\pgf@circ@avfont] {\pgf@circ@avminus}
                \fi
            \fi
        \fi
    \fi
}

%% Output routine
\def\pgf@circ@drawvoltage{% node name
    \pgfextra{ %WARNING: indentation is probably wrong
        \edef\pgfcircmathresult{\expandafter\pgf@circ@stripdecimals\pgf@circ@direction\pgf@nil}
        \ifnum\pgfcircmathresult >4 \ifnum\pgfcircmathresult <86
        \ifpgf@circuit@bipole@voltage@below
            \def\pgf@circ@bipole@voltage@label@anchor{north west}
        \else
            \def\pgf@circ@bipole@voltage@label@anchor{south east}
        \fi
        \fi\fi
        \ifnum\pgfcircmathresult >85 \ifnum\pgfcircmathresult <95
        \ifpgf@circuit@bipole@voltage@below
            \def\pgf@circ@bipole@voltage@label@anchor{west}
        \else
            \def\pgf@circ@bipole@voltage@label@anchor{east}
        \fi
        \fi\fi
        \ifnum\pgfcircmathresult >94 \ifnum\pgfcircmathresult <176
        \ifpgf@circuit@bipole@voltage@below
            \def\pgf@circ@bipole@voltage@label@anchor{south west}
        \else \def\pgf@circ@bipole@voltage@label@anchor{north east}
        \fi
        \fi\fi
        \ifnum\pgfcircmathresult >175 \ifnum\pgfcircmathresult <185
        \ifpgf@circuit@bipole@voltage@below
            \def\pgf@circ@bipole@voltage@label@anchor{south}
        \else\def\pgf@circ@bipole@voltage@label@anchor{north}
        \fi
        \fi\fi
        \ifnum\pgfcircmathresult >184 \ifnum\pgfcircmathresult <266
        \ifpgf@circuit@bipole@voltage@below
            \def\pgf@circ@bipole@voltage@label@anchor{south east}
        \else\def\pgf@circ@bipole@voltage@label@anchor{north west}
        \fi
        \fi\fi
        \ifnum\pgfcircmathresult >265 \ifnum\pgfcircmathresult <275
        \ifpgf@circuit@bipole@voltage@below
            \def\pgf@circ@bipole@voltage@label@anchor{east}
        \else \def\pgf@circ@bipole@voltage@label@anchor{west}
        \fi
        \fi\fi
        \ifnum\pgfcircmathresult >274 \ifnum\pgfcircmathresult <356
        \ifpgf@circuit@bipole@voltage@below
            \def\pgf@circ@bipole@voltage@label@anchor{north east}
        \else\def\pgf@circ@bipole@voltage@label@anchor{south west}
        \fi
        \fi\fi
        \ifnum\pgfcircmathresult >-1 \ifnum\pgfcircmathresult <5
        \ifpgf@circuit@bipole@voltage@below
            \def\pgf@circ@bipole@voltage@label@anchor{north}
        \else\def\pgf@circ@bipole@voltage@label@anchor{south}
        \fi
        \fi\fi
        \ifnum\pgfcircmathresult >355 \ifnum\pgfcircmathresult <361
        \ifpgf@circuit@bipole@voltage@below
            \def\pgf@circ@bipole@voltage@label@anchor{north}
        \else\def\pgf@circ@bipole@voltage@label@anchor{south}
        \fi
        \fi\fi

        \ifnum \ctikzvalof{mirror value}=-1
            \ifpgf@circuit@bipole@voltage@below
                \pgf@circuit@bipole@voltage@belowfalse
            \else
                \pgf@circuit@bipole@voltage@belowtrue
            \fi
        \fi

        \ifpgf@circuit@bipole@inverted
            \ifpgf@circuit@bipole@voltage@below
                \pgf@circuit@bipole@voltage@belowfalse
            \else

                \pgf@circuit@bipole@voltage@belowtrue
            \fi
        \fi

        \ifpgf@circuit@bipole@voltage@below
            \def\pgf@circ@bipole@voltage@label@where{-90}
        \else
            \def\pgf@circ@bipole@voltage@label@where{90}
        \fi


        \edef\pgf@temp{/tikz/circuitikz/bipoles/\pgfkeysvalueof{/tikz/circuitikz/bipole/kind}/voltage/european label distance}
        \pgfkeysifdefined{\pgf@temp}
        { \edef\eudist{\ctikzvalof{bipoles/\pgfkeysvalueof{/tikz/circuitikz/bipole/kind}/voltage/european label distance}} }
        { \edef\eudist{\ctikzvalof{voltage/european label distance}} }
        \edef\shiftv{\ctikzvalof{voltage/shift}}
        % adjust the label distance to the shift.
        \pgfmathsetmacro{\eudistplus}{\eudist+\shiftv}

        \pgfsetcornersarced{\pgfpointorigin}% do not use rounded corners!
    }%end pgfextra

    \ifpgf@circuit@bipole@isvoltage
        \pgf@circ@drawvoltagegenerator
    \else
        \pgf@circ@drawvoltagegeneric
    \fi

    %	(\ctikzvalof{bipole/name}.\pgf@circ@bipole@voltage@label@where) %Zeile sinnlos!?
    \ifpgf@circuit@bipole@voltage@straight
        coordinate (Vlab) at ($(pgfcirc@Vto)!0.5!(pgfcirc@Vfrom) $)
        node [anchor=\pgf@circ@bipole@voltage@label@anchor, inner sep=2pt]
        at (Vlab) { \pgf@circ@finallabels{voltage/label} }
    \else
        coordinate (Vlab) at ($(\ctikzvalof{bipole/name}.center) !
        \ifpgf@circuit@europeanvoltage
            \eudistplus
        \else
            \ctikzvalof{voltage/american label distance}
        \fi !
        (\ctikzvalof{bipole/name}.\pgf@circ@bipole@voltage@label@where)$)
        node [anchor=\pgf@circ@bipole@voltage@label@anchor, inner sep=2pt] at (Vlab) { \pgf@circ@finallabels{voltage/label} }
    \fi
}%end drawvoltages
\endinput
