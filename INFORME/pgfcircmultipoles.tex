% Copyright 2007-2009 by Massimo Redaelli
% Copyright 2019 by Romano Giannetti
%
% This file may be distributed and/or modified
%
% 1. under the LaTeX Project Public License and/or
% 2. under the GNU Public License.
%
% See the files gpl-3.0_license.txt and lppl-1-3c_license.txt for more details.

%%%%%%%%%%%%%%%%%%%%%%%%%%%%%%%%%%%%%%
%% Multipoles
%%%%%%%%%%%%%%%%%%%%%%%%%%%%%%%%%%%%%%

% counters for pin accounting...
\newcount\pgf@circ@count@c
\newcount\pgf@circ@count@b
\def\pgf@circ@dip@pin@shift{0.5}
\def\pgf@circ@qfp@pin@shift{0.25}

% derived from https://tex.stackexchange.com/a/146753/38080
% original author Mark Wibrow
% Thanks also to John Kormylo https://tex.stackexchange.com/a/372996/38080
% a lot of thanks to @marmot  for the un-rotation hint
% https://tex.stackexchange.com/a/473571/38080

% DIP (dual in line package) chips

\pgfdeclareshape{dipchip}{
    \savedanchor\centerpoint{%
        \pgf@x=-.5\wd\pgfnodeparttextbox%
        \pgf@y=-.5\ht\pgfnodeparttextbox%
        \advance\pgf@y by+.5\dp\pgfnodeparttextbox%
    }%
    \savedanchor\origin{\pgfpoint{0pt}{0pt}}
    \anchor{center}{\origin}
    \anchor{text}{\centerpoint}% to adjust text
    \saveddimen\height{%
        \pgfmathsetlength\pgf@x{((\pgfkeysvalueof{/tikz/circuitikz/multipoles/dipchip/num pins})
        *\pgfkeysvalueof{/tikz/circuitikz/multipoles/dipchip/pin spacing})*\pgf@circ@Rlen/2}%
    }%
    \saveddimen{\chipspacing}{\pgfmathsetlength\pgf@x{\pgf@circ@Rlen*\pgfkeysvalueof{/tikz/circuitikz/multipoles/dipchip/pin spacing}}}
    \saveddimen{\width}{\pgfmathsetlength\pgf@x{\pgf@circ@Rlen*\pgfkeysvalueof{/tikz/circuitikz/multipoles/dipchip/width}}}
    \saveddimen{\extshift}{\pgfmathsetlength\pgf@x{\pgf@circ@Rlen*\pgfkeysvalueof{/tikz/circuitikz/multipoles/external pins width}}}
    % standard anchors
    \savedanchor\northwest{%
        \pgfmathsetlength\pgf@y{0.5*((\pgfkeysvalueof{/tikz/circuitikz/multipoles/dipchip/num pins})
        *\pgfkeysvalueof{/tikz/circuitikz/multipoles/dipchip/pin spacing})*\pgf@circ@Rlen/2}%
        \pgfmathsetlength\pgf@x{-0.5*\pgf@circ@Rlen*\pgfkeysvalueof{/tikz/circuitikz/multipoles/dipchip/width}}
    }
    \anchor{dot}{\northwest
        \pgfmathsetlength\pgf@x{\pgf@x + 0.3*\chipspacing}
        \pgfmathsetlength\pgf@y{\pgf@y - 0.3*\chipspacing}
    }
    \anchor{nw}{\northwest}
    \anchor{ne}{\northwest\pgf@x=-\pgf@x}
    \anchor{se}{\northwest\pgf@x=-\pgf@x\pgf@y=-\pgf@y}
    \anchor{sw}{\northwest\pgf@y=-\pgf@y}
    \anchor{north west}{\northwest}
    \anchor{north east}{\northwest\pgf@x=-\pgf@x}
    \anchor{south east}{\northwest\pgf@x=-\pgf@x \pgf@y=-\pgf@y}
    \anchor{south west}{\northwest\pgf@y=-\pgf@y}
    \anchor{n}{\northwest\pgf@x=0pt }
    \anchor{e}{\northwest\pgf@x=-\pgf@x\pgf@y=0pt }
    \anchor{s}{\northwest\pgf@x=0pt\pgf@y=-\pgf@y}
    \anchor{w}{\northwest\pgf@y=0pt }
    \anchor{north}{\northwest\pgf@x=0pt }
    \anchor{east}{\northwest\pgf@x=-\pgf@x\pgf@y=0pt }
    \anchor{south}{\northwest\pgf@x=0pt\pgf@y=-\pgf@y}
    \anchor{west}{\northwest\pgf@y=0pt }
    % start drawing
    \backgroundpath{%
        \northwest
        \pgf@circ@res@up = \pgf@y
        \pgf@circ@res@down = -\pgf@y
        \pgf@circ@res@right = -\pgf@x
        \pgf@circ@res@left = \pgf@x
        \pgf@circ@res@step = \pgfkeysvalueof{/tikz/circuitikz/multipoles/dipchip/pin spacing}\pgf@circ@Rlen
        \pgf@circ@res@other = \pgfkeysvalueof{/tikz/circuitikz/multipoles/external pins width}\pgf@circ@Rlen
        \pgfscope% (for the line width)
        \pgfsetlinewidth{\pgfkeysvalueof{/tikz/circuitikz/multipoles/thickness}\pgflinewidth}
        \pgfpathrectanglecorners{\pgfpoint{-\width/2}{-\height/2}}{\pgfpoint{\width/2}{\height/2}}%
        \pgf@circ@draworfill
        %% upside mark
        \ifpgf@circuit@chip@topmark
            \pgfpathmoveto{\pgfpoint{0.2*\pgf@circ@res@left}{\pgf@circ@res@up}}
            \pgfpatharc{0}{180}{0.2*\pgf@circ@res@left}
        \fi
        \pgfusepath{stroke}%
        \pgfsetcolor{\pgfkeysvalueof{/tikz/circuitikz/color}}
        % Adding the pin number
        \ifpgf@circuit@chip@shownumbers
            \c@pgf@counta=\pgfkeysvalueof{/tikz/circuitikz/multipoles/dipchip/num pins}%
            \divide\c@pgf@counta by 2 \pgf@circ@count@b=\c@pgf@counta
            % thanks to @marmot: https://tex.stackexchange.com/a/473571/38080
            \ifpgf@circuit@chip@straightnumbers
                \pgfgettransformentries\a\b\temp\temp\temp\temp
                \pgfmathsetmacro{\rot}{-atan2(\b,\a)}
                \pgfmathtruncatemacro{\quadrant}{mod(4+int(360+(\rot+45)/90),4)}
            \else
                \pgfmathsetmacro{\rot}{0}
                \pgfmathsetmacro{\quadrant}{0}
            \fi
            \def\pgf@circ@strut{\vrule width 0pt height 1em depth 0.4em\relax}
            \def\mytext{\pgfkeysvalueof{/tikz/circuitikz/multipoles/font}\space\pgf@circ@strut\the\pgf@circ@count@c\space}
            \pgfmathloop%
            \ifnum\c@pgf@counta>0
                \ifcase\quadrant % rotation 0
                    % left
                    \pgf@circ@count@c=\c@pgf@counta
                    \pgftext[left,
                        at=\pgfpoint{\pgf@circ@res@left}{\pgf@circ@res@up+(\pgf@circ@dip@pin@shift-\the\c@pgf@counta)*\pgf@circ@res@step},
                        rotate=\rot]{\mytext}
                    % right
                    \pgf@circ@count@c=\numexpr2*\pgf@circ@count@b-\c@pgf@counta+1\relax
                    \pgftext[right,
                        at=\pgfpoint{\pgf@circ@res@right}{\pgf@circ@res@up+(\pgf@circ@dip@pin@shift-\the\c@pgf@counta)*\pgf@circ@res@step},
                        rotate=\rot]{\mytext}
                \or % rotation -90
                    % left
                    \pgf@circ@count@c=\c@pgf@counta
                    \pgftext[top,
                        at=\pgfpoint{\pgf@circ@res@left}{\pgf@circ@res@up+(\pgf@circ@dip@pin@shift-\the\c@pgf@counta)*\pgf@circ@res@step},
                        rotate=\rot]{\mytext}
                    % right
                    \pgf@circ@count@c=\numexpr2*\pgf@circ@count@b-\c@pgf@counta+1\relax
                    \pgftext[bottom,
                        at=\pgfpoint{\pgf@circ@res@right}{\pgf@circ@res@up+(\pgf@circ@dip@pin@shift-\the\c@pgf@counta)*\pgf@circ@res@step},
                        rotate=\rot]{\mytext}
                \or %rotation 180
                    % left
                    \pgf@circ@count@c=\c@pgf@counta
                    \pgftext[right,
                        at=\pgfpoint{\pgf@circ@res@left}{\pgf@circ@res@up+(\pgf@circ@dip@pin@shift-\the\c@pgf@counta)*\pgf@circ@res@step},
                        rotate=\rot]{\mytext}
                    % right
                    \pgf@circ@count@c=\numexpr2*\pgf@circ@count@b-\c@pgf@counta+1\relax
                    \pgftext[left,
                        at=\pgfpoint{\pgf@circ@res@right}{\pgf@circ@res@up+(\pgf@circ@dip@pin@shift-\the\c@pgf@counta)*\pgf@circ@res@step},
                        rotate=\rot]{\mytext}
                \or % rotation +90
                    % left
                    \pgf@circ@count@c=\c@pgf@counta
                    \pgftext[bottom,
                        at=\pgfpoint{\pgf@circ@res@left}{\pgf@circ@res@up+(\pgf@circ@dip@pin@shift-\the\c@pgf@counta)*\pgf@circ@res@step},
                        rotate=\rot]{\mytext}
                    % right
                    \pgf@circ@count@c=\numexpr2*\pgf@circ@count@b-\c@pgf@counta+1\relax
                    \pgftext[top,
                        at=\pgfpoint{\pgf@circ@res@right}{\pgf@circ@res@up+(\pgf@circ@dip@pin@shift-\the\c@pgf@counta)*\pgf@circ@res@step},
                        rotate=\rot]{\mytext}
                \fi
                \advance\c@pgf@counta-1\relax%
                \repeatpgfmathloop
            \fi
            \endpgfscope
            \ifdim\pgf@circ@res@other>0pt
            \pgfscope
                \pgfsetlinewidth{\pgfkeysvalueof{/tikz/circuitikz/multipoles/external pins thickness}\pgflinewidth}
                \c@pgf@counta=\pgfkeysvalueof{/tikz/circuitikz/multipoles/dipchip/num pins}%
                \divide\c@pgf@counta by 2 \pgf@circ@count@b=\c@pgf@counta
                \pgfmathloop%
                \ifnum\c@pgf@counta>0
                    \edef\padfrac{\pgfkeysvalueof{/tikz/circuitikz/multipoles/external pad fraction}}
                    \ifnum\padfrac>0
                        \pgf@circ@res@temp=\pgf@circ@res@step\divide\pgf@circ@res@temp by \padfrac
                        % left side pads
                        \pgfpathmoveto{\pgfpoint{\pgf@circ@res@left}{\pgf@circ@res@temp+\pgf@circ@res@up+(\pgf@circ@dip@pin@shift-\the\c@pgf@counta)*\pgf@circ@res@step}}
                        \pgfpathlineto{\pgfpoint{\pgf@circ@res@left-\pgf@circ@res@other}{\pgf@circ@res@temp+\pgf@circ@res@up+(\pgf@circ@dip@pin@shift-\the\c@pgf@counta)*\pgf@circ@res@step}}
                        \pgfpathlineto{\pgfpoint{\pgf@circ@res@left-\pgf@circ@res@other}{-\pgf@circ@res@temp+\pgf@circ@res@up+(\pgf@circ@dip@pin@shift-\the\c@pgf@counta)*\pgf@circ@res@step}}
                        \pgfpathlineto{\pgfpoint{\pgf@circ@res@left}{-\pgf@circ@res@temp+\pgf@circ@res@up+(\pgf@circ@dip@pin@shift-\the\c@pgf@counta)*\pgf@circ@res@step}}
                        % right side pads
                        \pgfpathmoveto{\pgfpoint{\pgf@circ@res@right}{\pgf@circ@res@temp+\pgf@circ@res@up+(\pgf@circ@dip@pin@shift-\the\c@pgf@counta)*\pgf@circ@res@step}}
                        \pgfpathlineto{\pgfpoint{\pgf@circ@res@right+\pgf@circ@res@other}{\pgf@circ@res@temp+\pgf@circ@res@up+(\pgf@circ@dip@pin@shift-\the\c@pgf@counta)*\pgf@circ@res@step}}
                        \pgfpathlineto{\pgfpoint{\pgf@circ@res@right+\pgf@circ@res@other}{-\pgf@circ@res@temp+\pgf@circ@res@up+(\pgf@circ@dip@pin@shift-\the\c@pgf@counta)*\pgf@circ@res@step}}
                        \pgfpathlineto{\pgfpoint{\pgf@circ@res@right}{-\pgf@circ@res@temp+\pgf@circ@res@up+(\pgf@circ@dip@pin@shift-\the\c@pgf@counta)*\pgf@circ@res@step}}
                    \else
                        % left side pins
                        \pgfpathmoveto{\pgfpoint{\pgf@circ@res@left}{\pgf@circ@res@up+(\pgf@circ@dip@pin@shift-\the\c@pgf@counta)*\pgf@circ@res@step}}
                        \pgfpathlineto{\pgfpoint{\pgf@circ@res@left-\pgf@circ@res@other}{\pgf@circ@res@up+(\pgf@circ@dip@pin@shift-\the\c@pgf@counta)*\pgf@circ@res@step}}
                        % right side pins
                        \pgfpathmoveto{\pgfpoint{\pgf@circ@res@right}{\pgf@circ@res@up+(\pgf@circ@dip@pin@shift-\the\c@pgf@counta)*\pgf@circ@res@step}}
                        \pgfpathlineto{\pgfpoint{\pgf@circ@res@right+\pgf@circ@res@other}{\pgf@circ@res@up+(\pgf@circ@dip@pin@shift-\the\c@pgf@counta)*\pgf@circ@res@step}}
                    \fi
                    \advance\c@pgf@counta by -1\relax%
                \repeatpgfmathloop
                \pgfusepath{stroke}
            \endpgfscope
            \fi
        }%
        % \pgf@sh@s@<name of the shape here> contains all the code for the shape
        % and is executed just before a node is drawn.
        \pgfutil@g@addto@macro\pgf@sh@s@dipchip{%
            % Start with the maximum pin number and go backwards.
            % If the anchor is undefined, create it. Otherwise stop.
            \c@pgf@counta=\pgfkeysvalueof{/tikz/circuitikz/multipoles/dipchip/num pins}%
            \divide\c@pgf@counta by 2 \pgf@circ@count@b=\c@pgf@counta
            \pgfmathloop%
            \ifnum\c@pgf@counta>0
                % left side, pins 1..npins/2
                % we will create two anchors per pin: the "normal one" like `pin 1` for the
                % electrical contact, and the "border one" like `bpin 1` for labels.
                % they will coincide if `external pins width` is set to 0.
                \expandafter\xdef\csname pgf@anchor@dipchip@bpin\space\the\c@pgf@counta\endcsname{%
                    \noexpand\pgf@circ@chippinanchorLB{\the\c@pgf@counta}%
                }
                \expandafter\xdef\csname pgf@anchor@dipchip@pin\space\the\c@pgf@counta\endcsname{%
                    \noexpand\pgf@circ@chippinanchorL{\the\c@pgf@counta}%
                }
                % right side
                \pgf@circ@count@c=\numexpr2*\pgf@circ@count@b-\c@pgf@counta+1\relax
                \expandafter\xdef\csname pgf@anchor@dipchip@bpin\space\the\pgf@circ@count@c\endcsname{%
                    \noexpand\pgf@circ@chippinanchorRB{\the\c@pgf@counta}%
                }%
                \expandafter\xdef\csname pgf@anchor@dipchip@pin\space\the\pgf@circ@count@c\endcsname{%
                    \noexpand\pgf@circ@chippinanchorR{\the\c@pgf@counta}%
                }%
                \advance\c@pgf@counta by -1\relax%
                \repeatpgfmathloop%
            }%
        }

% QFP (quad flat package) chips

\pgfdeclareshape{qfpchip}{
    \savedanchor\centerpoint{%
        \pgf@x=-.5\wd\pgfnodeparttextbox%
        \pgf@y=-.5\ht\pgfnodeparttextbox%
        \advance\pgf@y by+.5\dp\pgfnodeparttextbox%
    }%
    \savedanchor\origin{\pgfpoint{0pt}{0pt}}
    \anchor{center}{\origin}
    \anchor{text}{\centerpoint}% to adjust text
    \saveddimen\height{%
        \pgfmathsetlength\pgf@x{((\pgfkeysvalueof{/tikz/circuitikz/multipoles/qfpchip/num pins}+2)
        *\pgfkeysvalueof{/tikz/circuitikz/multipoles/qfpchip/pin spacing})*\pgf@circ@Rlen/4}%
    }%
    \saveddimen\width{%
        \pgfmathsetlength\pgf@x{((\pgfkeysvalueof{/tikz/circuitikz/multipoles/qfpchip/num pins}+2)
        *\pgfkeysvalueof{/tikz/circuitikz/multipoles/qfpchip/pin spacing})*\pgf@circ@Rlen/4}%
    }%
    \saveddimen{\chipspacing}{\pgfmathsetlength\pgf@x{\pgf@circ@Rlen*\pgfkeysvalueof{/tikz/circuitikz/multipoles/qfpchip/pin spacing}}}
    \saveddimen{\extshift}{\pgfmathsetlength\pgf@x{\pgf@circ@Rlen*\pgfkeysvalueof{/tikz/circuitikz/multipoles/external pins width}}}
    % standard anchors
    \savedanchor\northwest{%
        \pgfmathsetlength\pgf@y{0.5*((\pgfkeysvalueof{/tikz/circuitikz/multipoles/qfpchip/num pins}+2)
        *\pgfkeysvalueof{/tikz/circuitikz/multipoles/qfpchip/pin spacing})*\pgf@circ@Rlen/4}%
        \pgf@x=-\pgf@y
    }
    \anchor{dot}{\northwest
        \pgfmathsetlength\pgf@x{\pgf@x + 0.3*\chipspacing}
        \pgfmathsetlength\pgf@y{\pgf@y - 0.3*\chipspacing}
    }
    \anchor{nw}{\northwest}
    \anchor{ne}{\northwest\pgf@x=-\pgf@x}
    \anchor{se}{\northwest\pgf@x=-\pgf@x\pgf@y=-\pgf@y}
    \anchor{sw}{\northwest\pgf@y=-\pgf@y}
    \anchor{north west}{\northwest}
    \anchor{north east}{\northwest\pgf@x=-\pgf@x}
    \anchor{south east}{\northwest\pgf@x=-\pgf@x \pgf@y=-\pgf@y}
    \anchor{south west}{\northwest\pgf@y=-\pgf@y}
    \anchor{n}{\northwest\pgf@x=0pt }
    \anchor{e}{\northwest\pgf@x=-\pgf@x\pgf@y=0pt }
    \anchor{s}{\northwest\pgf@x=0pt\pgf@y=-\pgf@y}
    \anchor{w}{\northwest\pgf@y=0pt }
    \anchor{north}{\northwest\pgf@x=0pt }
    \anchor{east}{\northwest\pgf@x=-\pgf@x\pgf@y=0pt }
    \anchor{south}{\northwest\pgf@x=0pt\pgf@y=-\pgf@y}
    \anchor{west}{\northwest\pgf@y=0pt }
    % start drawing
    \backgroundpath{%
        \northwest
        \pgf@circ@res@up = \pgf@y
        \pgf@circ@res@down = -\pgf@y
        \pgf@circ@res@right = -\pgf@x
        \pgf@circ@res@left = \pgf@x
        \pgf@circ@res@step = \pgfkeysvalueof{/tikz/circuitikz/multipoles/qfpchip/pin spacing}\pgf@circ@Rlen
        \pgf@circ@res@other = \pgfkeysvalueof{/tikz/circuitikz/multipoles/external pins width}\pgf@circ@Rlen
        \pgfscope% (for the line width)
        \pgfsetlinewidth{\pgfkeysvalueof{/tikz/circuitikz/multipoles/thickness}\pgflinewidth}
        %% upside mark
        \ifpgf@circuit@chip@topmark
            \pgfpathmoveto{\pgfpoint{-\width/2}{\height/2-\pgf@circ@res@step/2}}
            \pgfpathlineto{\pgfpoint{-\width/2+\pgf@circ@res@step/2}{\height/2}}
        \else
            \pgfpathmoveto{\pgfpoint{-\width/2}{\height/2}}
        \fi
        %% rest of the shape
        \pgfpathlineto{\pgfpoint{\width/2}{\height/2}}
        \pgfpathlineto{\pgfpoint{\width/2}{-\height/2}}
        \pgfpathlineto{\pgfpoint{-\width/2}{-\height/2}}
        \pgfpathclose
        \pgf@circ@draworfill
        % Adding the pin number
        \pgfsetcolor{\pgfkeysvalueof{/tikz/circuitikz/color}}
        \ifpgf@circuit@chip@shownumbers
            \c@pgf@counta=\pgfkeysvalueof{/tikz/circuitikz/multipoles/qfpchip/num pins}%
            \divide\c@pgf@counta by 4 \pgf@circ@count@b=\c@pgf@counta
            % thanks to @marmot: https://tex.stackexchange.com/a/473571/38080
            \ifpgf@circuit@chip@straightnumbers
                \pgfgettransformentries\a\b\temp\temp\temp\temp
                \pgfmathsetmacro{\rot}{-atan2(\b,\a)}
                \pgfmathtruncatemacro{\quadrant}{mod(4+int(360+(\rot+45)/90),4)}
            \else
                \pgfmathsetmacro{\rot}{0}
                \pgfmathsetmacro{\quadrant}{0}
            \fi
            \def\pgf@circ@strut{\vrule width 0pt height 1em depth 0.4em\relax}
            \def\mytext{\pgfkeysvalueof{/tikz/circuitikz/multipoles/font}\space\pgf@circ@strut\the\pgf@circ@count@c\space}
            \pgfmathloop%
            \ifnum\c@pgf@counta>0
                \ifcase\quadrant % rotation 0
                    % left
                    \pgf@circ@count@c=\c@pgf@counta
                    \pgftext[left,
                        at=\pgfpoint{\pgf@circ@res@left}{\pgf@circ@res@up+(\pgf@circ@qfp@pin@shift-\the\c@pgf@counta)*\pgf@circ@res@step},
                        rotate=\rot]{\mytext}
                    % bottom
                    \pgf@circ@count@c=\numexpr\pgf@circ@count@b+\c@pgf@counta\relax
                    \pgftext[bottom,
                        at=\pgfpoint{\pgf@circ@res@left-(\pgf@circ@qfp@pin@shift-\the\c@pgf@counta)*\pgf@circ@res@step}{\pgf@circ@res@down},
                        rotate=\rot]{\mytext}
                    % right
                    \pgf@circ@count@c=\numexpr3*\pgf@circ@count@b-\c@pgf@counta+1\relax
                    \pgftext[right,
                        at=\pgfpoint{\pgf@circ@res@right}{\pgf@circ@res@up+(\pgf@circ@qfp@pin@shift-\the\c@pgf@counta)*\pgf@circ@res@step},
                        rotate=\rot]{\mytext}
                    % top
                    \pgf@circ@count@c=\numexpr3*\pgf@circ@count@b+\c@pgf@counta\relax
                    \pgftext[top,
                        at=\pgfpoint{\pgf@circ@res@right+(\pgf@circ@qfp@pin@shift-\the\c@pgf@counta)*\pgf@circ@res@step}{\pgf@circ@res@up},
                        rotate=\rot]{\mytext}
                \or % rotation -90
                    % left
                    \pgf@circ@count@c=\c@pgf@counta
                    \pgftext[top,
                        at=\pgfpoint{\pgf@circ@res@left}{\pgf@circ@res@up+(\pgf@circ@qfp@pin@shift-\the\c@pgf@counta)*\pgf@circ@res@step},
                        rotate=\rot]{\mytext}
                    % bottom
                    \pgf@circ@count@c=\numexpr\pgf@circ@count@b+\c@pgf@counta\relax
                    \pgftext[left,
                        at=\pgfpoint{\pgf@circ@res@left-(\pgf@circ@qfp@pin@shift-\the\c@pgf@counta)*\pgf@circ@res@step}{\pgf@circ@res@down},
                        rotate=\rot]{\mytext}
                    % right
                    \pgf@circ@count@c=\numexpr3*\pgf@circ@count@b-\c@pgf@counta+1\relax
                    \pgftext[bottom,
                        at=\pgfpoint{\pgf@circ@res@right}{\pgf@circ@res@up+(\pgf@circ@qfp@pin@shift-\the\c@pgf@counta)*\pgf@circ@res@step},
                        rotate=\rot]{\mytext}
                    % top
                    \pgf@circ@count@c=\numexpr3*\pgf@circ@count@b+\c@pgf@counta\relax
                    \pgftext[right,
                        at=\pgfpoint{\pgf@circ@res@right+(\pgf@circ@qfp@pin@shift-\the\c@pgf@counta)*\pgf@circ@res@step}{\pgf@circ@res@up},
                        rotate=\rot]{\mytext}
                \or %rotation 180
                    % left
                    \pgf@circ@count@c=\c@pgf@counta
                    \pgftext[right,
                        at=\pgfpoint{\pgf@circ@res@left}{\pgf@circ@res@up+(\pgf@circ@qfp@pin@shift-\the\c@pgf@counta)*\pgf@circ@res@step},
                        rotate=\rot]{\mytext}
                    % bottom
                    \pgf@circ@count@c=\numexpr\pgf@circ@count@b+\c@pgf@counta\relax
                    \pgftext[top,
                        at=\pgfpoint{\pgf@circ@res@left-(\pgf@circ@qfp@pin@shift-\the\c@pgf@counta)*\pgf@circ@res@step}{\pgf@circ@res@down},
                        rotate=\rot]{\mytext}
                    % right
                    \pgf@circ@count@c=\numexpr3*\pgf@circ@count@b-\c@pgf@counta+1\relax
                    \pgftext[left,
                        at=\pgfpoint{\pgf@circ@res@right}{\pgf@circ@res@up+(\pgf@circ@qfp@pin@shift-\the\c@pgf@counta)*\pgf@circ@res@step},
                        rotate=\rot]{\mytext}
                    % top
                    \pgf@circ@count@c=\numexpr3*\pgf@circ@count@b+\c@pgf@counta\relax
                    \pgftext[bottom,
                        at=\pgfpoint{\pgf@circ@res@right+(\pgf@circ@qfp@pin@shift-\the\c@pgf@counta)*\pgf@circ@res@step}{\pgf@circ@res@up},
                        rotate=\rot]{\mytext}
                \or % rotation +90
                    % left
                    \pgf@circ@count@c=\c@pgf@counta
                    \pgftext[bottom,
                        at=\pgfpoint{\pgf@circ@res@left}{\pgf@circ@res@up+(\pgf@circ@qfp@pin@shift-\the\c@pgf@counta)*\pgf@circ@res@step},
                        rotate=\rot]{\mytext}
                    % bottom
                    \pgf@circ@count@c=\numexpr\pgf@circ@count@b+\c@pgf@counta\relax
                    \pgftext[right,
                        at=\pgfpoint{\pgf@circ@res@left-(\pgf@circ@qfp@pin@shift-\the\c@pgf@counta)*\pgf@circ@res@step}{\pgf@circ@res@down},
                        rotate=\rot]{\mytext}
                    % right
                    \pgf@circ@count@c=\numexpr3*\pgf@circ@count@b-\c@pgf@counta+1\relax
                    \pgftext[top,
                        at=\pgfpoint{\pgf@circ@res@right}{\pgf@circ@res@up+(\pgf@circ@qfp@pin@shift-\the\c@pgf@counta)*\pgf@circ@res@step},
                        rotate=\rot]{\mytext}
                    % top
                    \pgf@circ@count@c=\numexpr3*\pgf@circ@count@b+\c@pgf@counta\relax
                    \pgftext[left,
                        at=\pgfpoint{\pgf@circ@res@right+(\pgf@circ@qfp@pin@shift-\the\c@pgf@counta)*\pgf@circ@res@step}{\pgf@circ@res@up},
                        rotate=\rot]{\mytext}
                \fi
                \advance\c@pgf@counta-1\relax%
                \repeatpgfmathloop
            \fi
            \endpgfscope
            \ifdim\pgf@circ@res@other>0pt
            \pgfscope
                \pgfsetlinewidth{\pgfkeysvalueof{/tikz/circuitikz/multipoles/external pins thickness}\pgflinewidth}
                \c@pgf@counta=\pgfkeysvalueof{/tikz/circuitikz/multipoles/qfpchip/num pins}%
                \divide\c@pgf@counta by 4 \pgf@circ@count@b=\c@pgf@counta
                \pgfmathloop%
                \ifnum\c@pgf@counta>0
                    \edef\padfrac{\pgfkeysvalueof{/tikz/circuitikz/multipoles/external pad fraction}}
                    \ifnum\padfrac>0
                        \pgf@circ@res@temp=\pgf@circ@res@step\divide\pgf@circ@res@temp by \padfrac
                        % left side pads
                        \pgfpathmoveto{\pgfpoint{\pgf@circ@res@left}{\pgf@circ@res@temp+\pgf@circ@res@up+(\pgf@circ@qfp@pin@shift-\the\c@pgf@counta)*\pgf@circ@res@step}}
                        \pgfpathlineto{\pgfpoint{\pgf@circ@res@left-\pgf@circ@res@other}{\pgf@circ@res@temp+\pgf@circ@res@up+(\pgf@circ@qfp@pin@shift-\the\c@pgf@counta)*\pgf@circ@res@step}}
                        \pgfpathlineto{\pgfpoint{\pgf@circ@res@left-\pgf@circ@res@other}{-\pgf@circ@res@temp+\pgf@circ@res@up+(\pgf@circ@qfp@pin@shift-\the\c@pgf@counta)*\pgf@circ@res@step}}
                        \pgfpathlineto{\pgfpoint{\pgf@circ@res@left}{-\pgf@circ@res@temp+\pgf@circ@res@up+(\pgf@circ@qfp@pin@shift-\the\c@pgf@counta)*\pgf@circ@res@step}}
                        % bottom side pads
                        \pgfpathmoveto{\pgfpoint{-\pgf@circ@res@temp+\pgf@circ@res@left-(\pgf@circ@qfp@pin@shift-\the\c@pgf@counta)*\pgf@circ@res@step}{\pgf@circ@res@down}}
                        \pgfpathlineto{\pgfpoint{-\pgf@circ@res@temp+\pgf@circ@res@left-(\pgf@circ@qfp@pin@shift-\the\c@pgf@counta)*\pgf@circ@res@step}{\pgf@circ@res@down-\pgf@circ@res@other}}
                        \pgfpathlineto{\pgfpoint{\pgf@circ@res@temp+\pgf@circ@res@left-(\pgf@circ@qfp@pin@shift-\the\c@pgf@counta)*\pgf@circ@res@step}{\pgf@circ@res@down-\pgf@circ@res@other}}
                        \pgfpathlineto{\pgfpoint{\pgf@circ@res@temp+\pgf@circ@res@left-(\pgf@circ@qfp@pin@shift-\the\c@pgf@counta)*\pgf@circ@res@step}{\pgf@circ@res@down}}
                        % right side pads
                        \pgfpathmoveto{\pgfpoint{\pgf@circ@res@right}{\pgf@circ@res@temp+\pgf@circ@res@up+(\pgf@circ@qfp@pin@shift-\the\c@pgf@counta)*\pgf@circ@res@step}}
                        \pgfpathlineto{\pgfpoint{\pgf@circ@res@right+\pgf@circ@res@other}{\pgf@circ@res@temp+\pgf@circ@res@up+(\pgf@circ@qfp@pin@shift-\the\c@pgf@counta)*\pgf@circ@res@step}}
                        \pgfpathlineto{\pgfpoint{\pgf@circ@res@right+\pgf@circ@res@other}{-\pgf@circ@res@temp+\pgf@circ@res@up+(\pgf@circ@qfp@pin@shift-\the\c@pgf@counta)*\pgf@circ@res@step}}
                        \pgfpathlineto{\pgfpoint{\pgf@circ@res@right}{-\pgf@circ@res@temp+\pgf@circ@res@up+(\pgf@circ@qfp@pin@shift-\the\c@pgf@counta)*\pgf@circ@res@step}}
                        % top side pads
                        \pgfpathmoveto{\pgfpoint{\pgf@circ@res@temp+\pgf@circ@res@right+(\pgf@circ@qfp@pin@shift-\the\c@pgf@counta)*\pgf@circ@res@step}{\pgf@circ@res@up}}
                        \pgfpathlineto{\pgfpoint{\pgf@circ@res@temp+\pgf@circ@res@right+(\pgf@circ@qfp@pin@shift-\the\c@pgf@counta)*\pgf@circ@res@step}{\pgf@circ@res@up+\pgf@circ@res@other}}
                        \pgfpathlineto{\pgfpoint{-\pgf@circ@res@temp+\pgf@circ@res@right+(\pgf@circ@qfp@pin@shift-\the\c@pgf@counta)*\pgf@circ@res@step}{\pgf@circ@res@up+\pgf@circ@res@other}}
                        \pgfpathlineto{\pgfpoint{-\pgf@circ@res@temp+\pgf@circ@res@right+(\pgf@circ@qfp@pin@shift-\the\c@pgf@counta)*\pgf@circ@res@step}{\pgf@circ@res@up}}
                    \else
                        % left side pins
                        \pgfpathmoveto{\pgfpoint{\pgf@circ@res@left}{\pgf@circ@res@up+(\pgf@circ@qfp@pin@shift-\the\c@pgf@counta)*\pgf@circ@res@step}}
                        \pgfpathlineto{\pgfpoint{\pgf@circ@res@left-\pgf@circ@res@other}{\pgf@circ@res@up+(\pgf@circ@qfp@pin@shift-\the\c@pgf@counta)*\pgf@circ@res@step}}
                        % bottom side pins
                        \pgfpathmoveto{\pgfpoint{\pgf@circ@res@left-(\pgf@circ@qfp@pin@shift-\the\c@pgf@counta)*\pgf@circ@res@step}{\pgf@circ@res@down}}
                        \pgfpathlineto{\pgfpoint{\pgf@circ@res@left-(\pgf@circ@qfp@pin@shift-\the\c@pgf@counta)*\pgf@circ@res@step}{\pgf@circ@res@down-\pgf@circ@res@other}}
                        % right side pins
                        \pgfpathmoveto{\pgfpoint{\pgf@circ@res@right}{\pgf@circ@res@up+(\pgf@circ@qfp@pin@shift-\the\c@pgf@counta)*\pgf@circ@res@step}}
                        \pgfpathlineto{\pgfpoint{\pgf@circ@res@right+\pgf@circ@res@other}{\pgf@circ@res@up+(\pgf@circ@qfp@pin@shift-\the\c@pgf@counta)*\pgf@circ@res@step}}
                        % top side pins
                        \pgfpathmoveto{\pgfpoint{\pgf@circ@res@right+(\pgf@circ@qfp@pin@shift-\the\c@pgf@counta)*\pgf@circ@res@step}{\pgf@circ@res@up}}
                        \pgfpathlineto{\pgfpoint{\pgf@circ@res@right+(\pgf@circ@qfp@pin@shift-\the\c@pgf@counta)*\pgf@circ@res@step}{\pgf@circ@res@up+\pgf@circ@res@other}}
                    \fi
                    \advance\c@pgf@counta-1\relax%
                \repeatpgfmathloop
                \pgfusepath{stroke}
            \endpgfscope
            \fi
        }%
        % \pgf@sh@s@<name of the shape here> contains all the code for the shape
        % and is executed just before a node is drawn.
        \pgfutil@g@addto@macro\pgf@sh@s@qfpchip{%
            % Start with the maximum pin number and go backwards.
            % If the anchor is undefined, create it. Otherwise stop.
            \c@pgf@counta=\pgfkeysvalueof{/tikz/circuitikz/multipoles/qfpchip/num pins}%
            \divide\c@pgf@counta by 4 \pgf@circ@count@b=\c@pgf@counta
            \pgfmathloop%
            \ifnum\c@pgf@counta>0
                % left side; 1..npins/4
                \expandafter\xdef\csname pgf@anchor@qfpchip@pin\space\the\c@pgf@counta\endcsname{%
                    \noexpand\pgf@circ@chippinanchorQL{\the\c@pgf@counta}%
                }
                \expandafter\xdef\csname pgf@anchor@qfpchip@bpin\space\the\c@pgf@counta\endcsname{%
                    \noexpand\pgf@circ@chippinanchorQLB{\the\c@pgf@counta}%
                }
                % bottom side;
                \pgf@circ@count@c=\numexpr2*\pgf@circ@count@b-\c@pgf@counta+1\relax
                \expandafter\xdef\csname pgf@anchor@qfpchip@pin\space\the\pgf@circ@count@c\endcsname{%
                    \noexpand\pgf@circ@chippinanchorQB{\the\c@pgf@counta}%
                }
                \expandafter\xdef\csname pgf@anchor@qfpchip@bpin\space\the\pgf@circ@count@c\endcsname{%
                    \noexpand\pgf@circ@chippinanchorQBB{\the\c@pgf@counta}%
                }
                % right side; 2*npins/4+1, 3*npins/4
                \pgf@circ@count@c=\numexpr3*\pgf@circ@count@b-\c@pgf@counta+1\relax
                \expandafter\xdef\csname pgf@anchor@qfpchip@pin\space\the\pgf@circ@count@c\endcsname{%
                    \noexpand\pgf@circ@chippinanchorQR{\the\c@pgf@counta}%
                }
                \expandafter\xdef\csname pgf@anchor@qfpchip@bpin\space\the\pgf@circ@count@c\endcsname{%
                    \noexpand\pgf@circ@chippinanchorQRB{\the\c@pgf@counta}%
                }
                % top side
                \pgf@circ@count@c=\numexpr3*\pgf@circ@count@b+\c@pgf@counta\relax
                \expandafter\xdef\csname pgf@anchor@qfpchip@pin\space\the\pgf@circ@count@c\endcsname{%
                    \noexpand\pgf@circ@chippinanchorQT{\the\c@pgf@counta}%
                }
                \expandafter\xdef\csname pgf@anchor@qfpchip@bpin\space\the\pgf@circ@count@c\endcsname{%
                    \noexpand\pgf@circ@chippinanchorQTB{\the\c@pgf@counta}%
                }
                \advance\c@pgf@counta-1\relax%
                \repeatpgfmathloop%
            }%
        }

%% anchors for DIP
\def\pgf@circ@chippinanchorR#1{%
  % When this macro is called,
  % \extshift, \height and \chipspacing will be defined.
    \pgfpoint{\width/2+\extshift}{\height/2+(\pgf@circ@dip@pin@shift-#1)*\chipspacing}%
}
\def\pgf@circ@chippinanchorL#1{%
  % When this macro is called,
  % \extshift, \height and \chipspacing will be defined.
    \pgfpoint{-\width/2-\extshift}{\height/2+(\pgf@circ@dip@pin@shift-#1)*\chipspacing}%
}
\def\pgf@circ@chippinanchorRB#1{%
  % When this macro is called,
  % \extshift, \height and \chipspacing will be defined.
    \pgfpoint{\width/2}{\height/2+(\pgf@circ@dip@pin@shift-#1)*\chipspacing}%
}
\def\pgf@circ@chippinanchorLB#1{%
  % When this macro is called,
  % \extshift, \height and \chipspacing will be defined.
    \pgfpoint{-\width/2}{\height/2+(\pgf@circ@dip@pin@shift-#1)*\chipspacing}%
}

%% anchors for QFP
\def\pgf@circ@chippinanchorQR#1{%
  % When this macro is called,
  % \extshift, \height and \chipspacing will be defined.
    \pgfpoint{\width/2+\extshift}{\height/2+(\pgf@circ@qfp@pin@shift-#1)*\chipspacing}%
}
\def\pgf@circ@chippinanchorQL#1{%
  % When this macro is called,
  % \extshift, \height and \chipspacing will be defined.
    \pgfpoint{-\width/2-\extshift}{\height/2+(\pgf@circ@qfp@pin@shift-#1)*\chipspacing}%
}
\def\pgf@circ@chippinanchorQT#1{%
  % When this macro is called,
  % \extshift, \height and \chipspacing will be defined.
    \pgfpoint{\width/2+(\pgf@circ@qfp@pin@shift-#1)*\chipspacing}{\height/2+\extshift}%
}
\def\pgf@circ@chippinanchorQB#1{%
  % When this macro is called,
  % \extshift, \height and \chipspacing will be defined.
    \pgfpoint{\width/2+(\pgf@circ@qfp@pin@shift-#1)*\chipspacing}{-\height/2-\extshift}%
}
\def\pgf@circ@chippinanchorQRB#1{%
  % When this macro is called,
  % \extshift, \height and \chipspacing will be defined.
    \pgfpoint{\width/2}{\height/2+(\pgf@circ@qfp@pin@shift-#1)*\chipspacing}%
}
\def\pgf@circ@chippinanchorQLB#1{%
  % When this macro is called,
  % \extshift, \height and \chipspacing will be defined.
    \pgfpoint{-\width/2}{\height/2+(\pgf@circ@qfp@pin@shift-#1)*\chipspacing}%
}
\def\pgf@circ@chippinanchorQTB#1{%
  % When this macro is called,
  % \extshift, \height and \chipspacing will be defined.
    \pgfpoint{\width/2+(\pgf@circ@qfp@pin@shift-#1)*\chipspacing}{\height/2}%
}
\def\pgf@circ@chippinanchorQBB#1{%
  % When this macro is called,
  % \extshift, \height and \chipspacing will be defined.
    \pgfpoint{\width/2+(\pgf@circ@qfp@pin@shift-#1)*\chipspacing}{-\height/2}%
}

